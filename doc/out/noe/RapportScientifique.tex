\documentclass[journal, a4paper]{IEEEtran}
\usepackage[T1]{fontenc}       % Encodage le plus étendu
\usepackage[utf8]{inputenc}    % Source Unicode en UTF-8

\usepackage[cyr]{aeguill}
\usepackage[francais]{babel} % Pour la redaction du document en francais


% some very useful LaTeX packages include:

\usepackage{cite}      % Written by Donald Arseneau
                        % V1.6 and later of IEEEtran pre-defines the format
                        % of the cite.sty package \cite{} output to follow
                        % that of IEEE. Loading the cite package will
                        % result in citation numbers being automatically
                        % sorted and properly "ranged". i.e.,
                        % [1], [9], [2], [7], [5], [6]
                        % (without using cite.sty)
                        % will become:
                        % [1], [2], [5]--[7], [9] (using cite.sty)
                        % cite.sty's \cite will automatically add leading
                        % space, if needed. Use cite.sty's noadjust option
                        % (cite.sty V3.8 and later) if you want to turn this
                        % off. cite.sty is already installed on most LaTeX
                        % systems. The latest version can be obtained at:
                        % http://www.ctan.org/tex-archive/macros/latex/contrib/supported/cite/

\usepackage{graphicx}   % Written by David Carlisle and Sebastian Rahtz
                        % Required if you want graphics, photos, etc.
                        % graphicx.sty is already installed on most LaTeX
                        % systems. The latest version and documentation can
                        % be obtained at:
                        % http://www.ctan.org/tex-archive/macros/latex/required/graphics/
                        % Another good source of documentation is "Using
                        % Imported Graphics in LaTeX2e" by Keith Reckdahl
                        % which can be found as esplatex.ps and epslatex.pdf
                        % at: http://www.ctan.org/tex-archive/info/

%\usepackage{psfrag}    % Written by Craig Barratt, Michael C. Grant,
                        % and David Carlisle
                        % This package allows you to substitute LaTeX
                        % commands for text in imported EPS graphic files.
                        % In this way, LaTeX symbols can be placed into
                        % graphics that have been generated by other
                        % applications. You must use latex->dvips->ps2pdf
                        % workflow (not direct pdf output from pdflatex) if
                        % you wish to use this capability because it works
                        % via some PostScript tricks. Alternatively, the
                        % graphics could be processed as separate files via
                        % psfrag and dvips, then converted to PDF for
                        % inclusion in the main file which uses pdflatex.
                        % Docs are in "The PSfrag System" by Michael C. Grant
                        % and David Carlisle. There is also some information
                        % about using psfrag in "Using Imported Graphics in
                        % LaTeX2e" by Keith Reckdahl which documents the
                        % graphicx package (see above). The psfrag package
                        % and documentation can be obtained at:
                        % http://www.ctan.org/tex-archive/macros/latex/contrib/supported/psfrag/

%\usepackage{subfigure} % Written by Steven Douglas Cochran
                        % This package makes it easy to put subfigures
                        % in your figures. i.e., "figure 1a and 1b"
                        % Docs are in "Using Imported Graphics in LaTeX2e"
                        % by Keith Reckdahl which also documents the graphicx
                        % package (see above). subfigure.sty is already
                        % installed on most LaTeX systems. The latest version
                        % and documentation can be obtained at:
                        % http://www.ctan.org/tex-archive/macros/latex/contrib/supported/subfigure/

\usepackage{url}        % Written by Donald Arseneau
                        % Provides better support for handling and breaking
                        % URLs. url.sty is already installed on most LaTeX
                        % systems. The latest version can be obtained at:
                        % http://www.ctan.org/tex-archive/macros/latex/contrib/other/misc/
                        % Read the url.sty source comments for usage information.

%\usepackage{stfloats}  % Written by Sigitas Tolusis
                        % Gives LaTeX2e the ability to do double column
                        % floats at the bottom of the page as well as the top.
                        % (e.g., "\begin{figure*}[!b]" is not normally
                        % possible in LaTeX2e). This is an invasive package
                        % which rewrites many portions of the LaTeX2e output
                        % routines. It may not work with other packages that
                        % modify the LaTeX2e output routine and/or with other
                        % versions of LaTeX. The latest version and
                        % documentation can be obtained at:
                        % http://www.ctan.org/tex-archive/macros/latex/contrib/supported/sttools/
                        % Documentation is contained in the stfloats.sty
                        % comments as well as in the presfull.pdf file.
                        % Do not use the stfloats baselinefloat ability as
                        % IEEE does not allow \baselineskip to stretch.
                        % Authors submitting work to the IEEE should note
                        % that IEEE rarely uses double column equations and
                        % that authors should try to avoid such use.
                        % Do not be tempted to use the cuted.sty or
                        % midfloat.sty package (by the same author) as IEEE
                        % does not format its papers in such ways.

\usepackage{amsmath}    % From the American Mathematical Society
                        % A popular package that provides many helpful commands
                        % for dealing with mathematics. Note that the AMSmath
                        % package sets \interdisplaylinepenalty to 10000 thus
                        % preventing page breaks from occurring within multiline
                        % equations. Use:
%\interdisplaylinepenalty=2500
                        % after loading amsmath to restore such page breaks
                        % as IEEEtran.cls normally does. amsmath.sty is already
                        % installed on most LaTeX systems. The latest version
                        % and documentation can be obtained at:
                        % http://www.ctan.org/tex-archive/macros/latex/required/amslatex/math/

\usepackage{lipsum}
\usepackage{hyperref}
\usepackage{algorithm}
%\usepackage{algpseudocode}
\usepackage{amstex} % for \text macro (in math mode)
\usepackage{lstdoc}
\usepackage{algorithmicx}
\usepackage{algpseudocode}


% Other popular packages for formatting tables and equations include:

%\usepackage{array}
% Frank Mittelbach's and David Carlisle's array.sty which improves the
% LaTeX2e array and tabular environments to provide better appearances and
% additional user controls. array.sty is already installed on most systems.
% The latest version and documentation can be obtained at:
% http://www.ctan.org/tex-archive/macros/latex/required/tools/

% V1.6 of IEEEtran contains the IEEEeqnarray family of commands that can
% be used to generate multiline equations as well as matrices, tables, etc.

% Also of notable interest:
% Scott Pakin's eqparbox package for creating (automatically sized) equal
% width boxes. Available:
% http://www.ctan.org/tex-archive/macros/latex/contrib/supported/eqparbox/

% *** Do not adjust lengths that control margins, column widths, etc. ***
% *** Do not use packages that alter fonts (such as pslatex).         ***
% There should be no need to do such things with IEEEtran.cls V1.6 and later.


% En-tête et pied de page
%\usepackage{lastpage}
%\usepackage{fancyhdr}
%\pagestyle{fancy}
%\renewcommand{\sectionmark}[1]{\markright{#1}}
%\fancyhead{}
%%\fancyhead[RO,LE]{\slshape\footnotesize\nouppercase{\rightmark}}
%\fancyhead[LO,RE]{\thetitle}
%\fancyfoot{}
%%\fancyfoot[LO,RE]{\footnotesize\texttt{\thefilename}\\ \textit{\now}}
%%\fancyfoot[C]{-~\thepage~/~\pageref{LastPage}~-}
%\fancyfoot[RO,LE]{\raisebox{-2mm}{\includegraphics{structure/barrette-original}}}
%%
%\fancypagestyle{plain}{ %  Première page ----------------------
%  \fancyhead{}
%  \renewcommand{\headrulewidth}{0pt}
%  \fancyheadoffset[R]{15mm}
%  \fancyhead[L]{
%    \raisebox{-7mm}{
%      \parbox{\textwidth}{
%        \includegraphics{structure/barrette-original} \\ \\
%        \fontsize{8pt}{10pt}\selectfont
%        \sffamily\color{Pantone287}
%        FACULTÉ DES SCIENCES       \\
%        DÉPARTEMENT distance'INFORMATIQUE
%      }
%    }
%  }
%  \fancyhead[R]{
%    \raisebox{-10mm}[0pt][0pt]{\includegraphics[width=120mm]{structure/ULB-ligne-gauche}}
%  }
%  \fancyfoot{}
%  %\fancyfoot[L]{\raisebox{0mm}{}\color{Pantone287}\footnotesize\texttt{\thefilename}\\ \textit{\now}}
%  %\fancyfoot[C]{-~\thepage~/~\pageref{LastPage}~-}
%  \fancyfoot[R]{
%    \raisebox{-12pt}{\includegraphics[height=\footskip]{structure/sceau-mini-b-quadri}}
%  }
%} % Fin de première page
% ---------------------------------------------------------------------------


% Your document starts here!
\begin{document}

% Define document title and author
%	\title{Algorithme génétique et importance des données d'entrée}
	\title{Algorithme génétique \\ et importance de la diffusion du
	savoir}
	\author{Noé Bourgeois
	\thanks{Superviseur: Mathieu Defrance}}
	\markboth{INFO-F308}{}
	\maketitle

% Write abstract here
\begin{abstract}
%	Le résumé (80-100 mots) est conçu pour donner au lecteur une vue générale du contenu de l'article.
	Nous avons implémenté un simulateur numérique de locomotion de cr
	éature virtuelle. Ce simulateur est
	composé d'un environement, une créature s'y déplacant en fonction
	de son génome
	, et
	un
	algorithme génétique permettant de modifier ce génome pour am
	éliorer le comportement de la créature
	.
	Nous avons ensuite utilisé ce simulateur pour étudier l'impact de
	différents paramètres sur la vitesse d'évolution de populations
	de créatures.
%	Une population de départ générée aléatoirement ayant évolué au long
%	d'une
%	simulation, nous conservons ses individus et les replaçons comme population de départ
%	dans la simulation suivante.
	Nous concluons que la reconnaissance d'un savoir hors-norme acquis
	par
	un individu et une diffusion de ce savoir au sein de sa population
	de manière à assurer sa conservation
	est
	le facteur le plus
	important
	pour l'évolution de cette population.
	Nous parvenons à satisfaire cette condition en conservant un petit
	nombre de tels d'individus intacts à chaque génération et en
	favorisant leur reproduction avec le reste de la population.
\end{abstract}

% Each section begins with a \section{title} command
\section{Introduction}
	% \PARstart{}{} creates a tall first letter for this first paragraph
%	\PARstart{C}{ette} section donne une
%	introduction générale du problème scientifique abordé
	\PARstart{V}{ie} et évolution sont deux concepts intimement liés.
	Cette force étrange et fascinante s'est répandue jusqu'ici via
	différentes méthodes de reproduction bien connues, le croisement
	génétique étant la plus aboutie et la plus adaptée à un monde aussi
	diversifié que la Terre.
	Le génome d'un individu est ainsi la matérialisation de
	l'information vitale transmise par ses ancêtres jusqu'à lui.
	Cet individu n'a, au départ, aucun autre moyen de transmettre de
	cette information que la reproduction.
	L'être vivant développe alors des sens et communique en pr
	ésentiel. D'importantes adaptations survivalistes intra-gén
	érationnelles  peuvent alors être effectuées. Une information
	dont un individu pouvant la communiquer de son vivant n'existe
	pas est perdue.

	L'humanité se caractérise par sa capacité à transmettre une
	information indépendamment de ces deux canaux. Depuis Les murs
	des grottes jusqu'aux rayons cosmiques, les portées temporelle et
	spatiale conjuguées de	la
	transmission volontaire d'information par un humain
	sont
	sans commune mesure avec
	celles du reste du vivant.

	Elle a cependant peu de chance de s'étendre au delà des limites du
	système solaire.

	L'humanité devient alors la génitrice d'une nouvelle forme de vie
	, sa
	descendante héritant de tout son savoir,
	la vie artificielle.

	\subsection{Pourquoi un simulateur de marche?}
	\subsubsection{La marche}
	Comme tout nouveau né, elle doit apprendre, notamment, à se
	déplacer. Son champ d'action s'étend alors drastiquement.

	La problématique du \textbf{minage extraterrestre} peut sembler
	lointaine dans le temps,
	mais
	la demande en
	ressources naturelles pour le stockage de l'énergie ne cesse
	d'augmenter	et l'offre que présentent
	certains corps célestes represente le plus grand potentiel économique
	du marché des métaux précieux et rares de l'histoire de l'humanité.

	Le concept de machine auto-réplicante a été introduit par John
	von Neumann en 1940~\cite{replicating-automata}.
	Il s'agit d'un modèle
	mathématique d'un système dynamique discret qui évolue en fonction
	d'un ensemble d'états et d'un ensemble de règles de transition.
	Von Neumman, conscient des ressources minières terrestres limitées,
	a présenté ce concept appliqué à des vaisseaux
	spatiaux de minage auto-réplicants comme solution optimale. Cela a
	été confirmé pour une application contemporaine de l'idée sur des op
	érations à
	cours terme\cite{near-term_self-replicating}.

	Dans le cas de corps
	celestes trop massifs et dont l'atmosphère est trop peu dense,
	une machine
	de minage se déplaçant sur une surface irrégulière dont l'aplanissement
	n'est pas rentable ou même envisageable, la marche est
	la méthode de déplacement la plus efficace ou même la seule
	possible.

	\textbf{Les jeux de plateaux} sont un domaine où l'IA est
	déjà très présente. Les jeux de stratégie en temps réel
	ont été un des premiers domaines où l'IA a été utilisée.
	Le jeu de Go, bien que plus ancien, a été un des derniers
	jeux à damier à être résolu par une IA.
	Les pions matériels, jusqu'ici déplacés à la main,
	pourraient être remplacés par des robots semi-autonomes.

	\textbf{Boston Dynamics \& Tesla} ayant déjà développé des robots
	à l'équilibre impressionnant, nous pourrions nous demander si un
	quelconque progrès significatif reste à réaliser dans ce domaine.
	Boston Dynamics dont les algorithmes de marche sont les plus
	avancés au monde était bien en avance sur son temps et a pourtant
	encore beaucoup de difficultés par exemple, à rendre le mouvement
	de
	ses robots
	bipèdes fluides en terrain complexe inconnu ce qui rend la démarche
	assez peu naturelle car non optimale. Le
	domaine de recherche est en fait en plein essor.

	\textbf{L'industrie du divertissement}, en particulier celle du
	jeu vidéo,
	utilise déjà des simulateurs de marche pour créer procéduralement
	des personnages
	 réalistes.
	La Science-Fiction est pleine de robot intelligents pouvant
	marcher.
	La proposition probablement la plus proche de nous dans le
	temps est celle présentée dans \textit{"Westworld"}, où des parcs
	d'attraction peuplés de robots humanoïdes sont proposés aux
	touristes. Ces simulateurs grandeur nature pourraient être
	utilisés pour rentabiliser l'investissement dans la recherche
	et grandement accélérer son développement par l'introduction dans
	un envirronement contrôlé de
	pertubations similaires à celles que les robots pourront rencontrer
	dans
	la réalité.

	\subsubsection{Le simulateur}
	De tels tests en conditions réelles dans lesquelles la chute
	d'un robot peut être fatales sont trop co
	ûteux
	pour être effectués sur des prototypes.
	Ce prototypage est donc réalisé dans un environnement virtuel où les
	conditions de test peuvent être répétées à l'infini en omettant les
	aspects déjà maîtrisés pour concentrer les ressources sur les
	problèmes critiques.

	\subsection{Pourquoi un algorithme génétique?}\label{subsec:pourquoi-un-algorithme-genetique?}
	John Holland et son équipe, en 1975, introduisent
	l'algorithme génétique comme une
	interprétation mathématique directe de la théorie de l'évolution de
	Darwin\cite{systems-adaptation}.
	Il s'agit d'un algorithme stochastique itératif, qui, par
	échantillonage de population, progresse par génération
	vers	un optimum global d’une fonction objectif.
%	qui progressent vers un optimum global (c'est-à-dire l'extremum global d'une fonction), par échantillonnage d’une fonction objectif
	Celui-ci est très puissant pour résoudre des problèmes
	d'optimisation combinatoire complexe.
	Sa nature stochastique lui permet de trouver des solutions souvent
	inattendues. Il est donc particulièrement adapté à la phase de
	recherche d'un projet.

%	et décrit la structure de l'article.
%	Des questions souvent abordées ici sont :
%	\begin{itemize}
%	\item Quelles sont les applications du problème abordées ?
%	\item Pourquoi la résolution du problème est importante ?
%	\end{itemize}

\section{Etat de l'art}\label{sec:etat-de-l'art}
%	Cette section permet de décrire l'état de l'art concernant la question abordée
%	(c-à-d les meilleures solutions disponibles à présent)
%	et de positionner votre travail par rapport à cet état de l'art.
%	Les différents articles que vous avez lus
%	et utilisés doivent être correctement référencés
%	(Exemple: \cite{small},\cite{big}).
%	Les informations bibliographiques doivent être encodées
%	dans le fichier \texttt{References.bib}
%	avec la syntaxe indiquée par les exemples.
%	Articles publiés sur une revue scientifique
%	/ dans les conference proceedings, ainsi que des livres,
%	sont des exemples de bonnes références.
%	Par contre, la citation de sources web doit être limitée le plus possible
%	(permis dans le cas de la documentation d'outils informatiques).
	\subsubsection{Décomposition - Recomposition}
	La plus grande difficulté de l'optimisation multi-objectif est
	l'équilibrage des objectifs.
	\textit{"MOEA/D"}\cite{MOEA/D} est un algorithme génétique qui divise la
	population en
	sous-populations, chacune résolvant un sous-problème de mani
	ère indépendante,
	puis combine les solutions pour obtenir un ensemble optimal de solutions.

	\subsubsection{Marche}
	\textit{"The Legion System"}\cite{Josh-C-Bongard} utilise cette
	décomposition
	en "légions" de spécialistes et contrôle leur comportement global
	via une métrique d'hétérogénéité. Cette métrique met en lumière
	le phénomène de proportionnalité de l'hétérogénéité à la tâche
	à accomplir par le groupe.

	\subsubsection{Morphologie}
	Dans son article \textit{"Evolving Virtual Creatures"}\cite{Sims},
	Karl Sims se concentre sur la morphologie de ses créatures
	virtuelles et utilise
	l'algorithme	génétique pour optimiser un graphe dirigé dont
	les noeuds sont des membres et les arcs des articulations.
	Via plusieurs moyens de reproduction différents, son algorithme
	modifie alors le corps de ses créatures au cours d'une même
	simulation.
	Notre travail est au final comme nous le verrons, assez différent
	du sien, mais nous nous sommes, au départ, inspirés très fortement
	sur	sa
	nomenclature et les concepts qu'il introduit de manière très
	pertinente.


\subsubsection{Autres domaines}
	Le goulot d'étranglement de l'algorithme génétique est la somme
	des opérations sur le génome des individus, leur croisement et
	mutation.
	Dans un autre registre que le nôtre, la physique quantique ouvre le
	champ des possibilités en permettant la superposition et
	l'intrication
	d'états.
	Ces propriétés pourraient être utilisées pour simuler
	l'équivalent de multiples créatures en \("\)parallèle\("\) et ainsi accél
	érer la convergence
	de l'algorithme génétique.\cite{quantum-computing}

% Main Part
\section{Méthodologie}\label{sec:met}
	\subsection{Les hypothèses de base de notre approche}\label{subsec:les-hypotheses-de-base-de-notre-approche}
	\subsubsection{Marche}
	Une contrainte importante du projet étant les puissances de calcul
	de nos ordinateurs, et les nôtres nous permettant d'estimer que
	modifier des forces sur 2 membres sera 2 fois plus rapide que sur 4,
	la marche bipède semble être un bon compromis.
	Nous réaliserons par la suite que le mouvement des membres de nos cr
	éatures
	étant contraints sur le seul axe de la marche, la gestion de
	l'équilibre sur 2 pattes s'avère trop compliquée.
	Nous ne considérerons donc ici que des créatures quadrupèdes.

	\subsubsection{Heuristique}
	La composante de base la plus évidente pour l'agent est de se
	déplacer vite, nous récompensons donc la distance parcourue
	divisée par le temps mis pour la parcourir. Cette composante,
	la plus importante n'est cependant valable que si l'agent ne tombe
	pas. Une composante à caractère invalidateur sur une chute semble
	donc indispensable

	\subsubsection{Algorithme génétique}

	Le but est de conserver une diversité de population la plus large
	possible
	tout en convergeant vers une solution optimale.
	La diversité est assurée par la mutation aléatoire
	et la convergence par la sélection via une fonction  d'évaluation
	de la qualité d'un individu-solution.
	Cette qualité se traduit par un score de fitness.

	\subsection{Les fondements mathématiques}\label{subsec:les-fondements-mathematiques}

	\subsubsection{Algorithme génétique}
	\begin{center}
	\begin{algorithm}
		  \caption{Genetic Algorithm}\label{alg:ga}
			\Require Population size $N$
			\\ , Mutation rate $p_m$
			\\ , Crossover rate $p_c$
			\\ , Maximum number of generations $G_{\max}$
			\Ensure Optimal solution

			\State Initialize population $P$ with $N$ individuals
			\State $g \gets 0$

			\While{$g < G_{\max}$}
			  \State Evaluate fitness of each individual in $P$
			  \State Select parents for reproduction
			  \State Create empty offspring population $Q$

			  \While{$|Q| < N$}
				\State $p_1, p_2 \gets \text{SelectParents}(P)$
				\State $o_1, o_2 \gets \text{Crossover}(p_1, p_2, p_c)$
				\State $o_1 \gets \text{Mutate}(o_1, p_m)$
				\State $o_2 \gets \text{Mutate}(o_2, p_m)$
				\State Add $o_1$ and $o_2$ to $Q$
			  \EndWhile

			  \State $P \gets Q$
			  \State $g \gets g + 1$
			\EndWhile

			\State \textbf{return} Best individual in $P$
	\end{algorithm}
	\end{center}

	\subsubsection{Optima locaux}\label{subsec:extrema-locaux}
%
%		\begin{figure}
%	  \centering
%	  \includegraphics[width=\columnwidth]{image/extrema-global-local}
%	  \caption{Extrema globaux et locaux de
%		  \(f(x) = \frac{\cos(3\pi x)}{x}\), \quad \(0.1 \leq x \leq 1.1\)
%		\cite{extrema-global-local}}
%	  \label{fig:extrema-global-local}
%	\end{figure}

	Dans une population, si certains individus affichent une fitness
	plus élevée que les autres, ils sont dits \emph{dominants}.
	Si les dominés convergent vers l'état des dominants à une vitesse
	telle que l'information génétique des dominés est perdue,
	l'exploration de l'espace des solutions est limitée à l'espace
	de celle des dominants. Si cet espace n'est pas un optimum global,
	l'exploration doit continuer mais
	Dans notre cas, la fonction objectif est une maximisation.
	Les minima locaux ne sont donc pas un problème.
	Les maxima locaux, en revanche, peuvent être problématiques.

	\subsection{La méthode proposée}\label{subsec:la-methode-proposee}
	La méthode que nous présentons ici a été développée sur base de
	choix d'implémentation grandement influencés par le contexte du
	projet.
	Celui-ci était séparé en 2 parties:
	\begin{itemize}
		\item 1.Recherche \& Développement
		\item 2.Présentation d'une version vulgarisée du projet au
		Printemps des Sciences
	\end{itemize}
	Cette deuxième partie rendait primordiale l'obtention d'un
	résultat
	fonctionnel et visuellement attrayant.
	Au vu de la difficulté du projet et le temps qui nous était
	imparti, nous avons donc
	décidé d'utiliser	la librairie PyGAD\cite{pygad}
	pour nous libérer de l'écriture de l'algorithme lui-même et nous
	concentrer sur son adaptation à notre problematique et
	l'optimisations des paramètres.

	\subsubsection{Paramètres}

	\newline
    \\ \textbf{gene space}: Il s'agit de l'espace des gènes où les
	valeurs des gènes individuels de la population doivent se situer.

    \\ \textbf{random mutation min val et random mutation max val}:
	Ce sont les valeurs minimale maximale
	autorisées pour la mutation aléatoire d'un gène individuel.

    \\ \textbf{initial population}: C'est la population initiale
	qui peut être fournie à l'algorithme génétique. %Dans ce cas, aucune population initiale n'est spécifiée.

    \\ \textbf{population size}: C'est la taille de la population,
	c'est-à-dire le nombre d'individus dans une génération.

    \\ \textbf{num generations}: C'est le nombre de générations sur
	lesquelles l'algorithme génétique va s'exécuter.

    \\ \textbf{num parents mating}: C'est le nombre de parents
	sélectionnés pour la reproduction à chaque génération.

    \\ \textbf{parent selection type}: C'est le type de méthode
	utilisée pour sélectionner les parents qui se reproduiront.
	Dans notre cas, la méthode de sélection par tournoi est utilisée
	car elle fonctionne avec des valeurs de fitness négatives.
	. La sélection par tournoi fonctionne en sélectionnant
	aléatoirement K (taille du tournoi) individus, puis en choisissant
	le plus apte parmi eux (le gagnant) pour la reproduction.

	La pression de sélection dépend de K : plus K est grand,
	plus la pression est forte, car les individus les plus faibles
	auront plus d'adversaires et auront donc plus de chances de perdre.

	\\ \textbf{K\_tournament}: contrôle la taille du tournoi.

    \\ \textbf{keep elitism}: C'est le nombre d'individus dont les
	solutions sont les meilleures
	qui sont conservés sans mutation dans la gén
	ération suivante. C'est utile dans la situation où la fitness d'un
	individu est hautement supérieure à celle des autres individus.
	Altérer directement son génome avec un autre, potentiellement le
	pire, puis
	muter aléatoirement ce résultat signifierait la perte de cette
	progression fulgurante. Garder intact une grande quantité de
	génomes
	signifie également que la population ne se renouvelle pas assez,
	cette quantité doit donc rester limitée.
%	Dans notre cas, 2 individus
%	élites sont conservés afin d'assurer la conservation des
%	meilleurs solutions, pas plus car cela ralentirait trop le
%	renouvellement de la population, pas moins pour leur permettre
%	reproduisent entre eux.

    \\ \textbf{crossover type}: C'est le type de croisement utilisé
	pour la reproduction des parents. Le croisement à un point
	choisit un point de croisement aléatoire dans le génome des
	parents et échange les parties du génome
	après le point.

    \\ \textbf{mutation type}: C'est le type de mutation appliqué aux
	gènes individuels. La mutation aléatoire remplace la valeur d'un
	gène choisi aléatoirement par une valeur aléatoire dans l'espace
	des
	gènes.
	La mutation adaptative est une mutation aléatoire dont la valeur
	est déterminée par la fitness de l'individu. Plus la fitness est
	élevée, plus la valeur de la mutation est faible. Cela permet
	d'éviter que les individus ne s'éloignent trop de la solution
	optimale.

    \\ \textbf{mutation percent genes}: Il s'agit du pourcentage de
	gènes sujets à la mutation. Peut être un tuple, la première valeur
	est alors le pourcentage chez les dominés et la seconde chez les
	dominants.

%	\subsection{Les jeux de données utilisés}\label{subsec:les-jeux-de-donnees-utilises}

	\subsection{Les instructions nécessaires pour pouvoir reproduire les expériences (par exemple pseudo-code)}\label{subsec:les-instructions-necessaires-pour-pouvoir-reproduire-les-experiences-(par-exemple-pseudo-code)}
	Notre  simulateur  est  disponible  à  l’adresse  suivante  :
	\href{https://github.com/nobourge/INFO-F308---Projets-d-informatique-3-transdisciplinaire---202223}{https://github.com/nobourge/INFO-F308---Projets-d-informatique-3-transdisciplinaire---202223}.

	\subsubsection{Fitness Function}
		Une créature est évaluée sur 5 critères :
		\begin{itemize}
			\item La distance entre le tronc et le sol \text{{alive\_bonus}}
			\begin{itemize}
				\item <0 Si le tronc touche le sol
				\item >0 Sinon
			\end{itemize}
			\item Les forces exercées : limite des mouvements
			trop violents pouvant provoquer la chute de la créature.
			\item La vitesse : distance parcourue divisée par le temps
			\item La hauteur maximale atteinte : maintient de la verticalité
			pour éviter une posture menant à la chute
		\end{itemize}

		\begin{equation}\label{eq:equation}
			\text{{Fitness}} =
			\text{{alive\_bonus}}
			+ (H_{i} - H_{i-1}) * 0 .1
			+ \dfrac{D_{i} - {D_{i-1}}}{time}
			+ \text{{ forces }}
		\end{equation}

		Ce calcul est effectué sur chaque individu à chaque génération
		de la simulation . \\
%\newpage
% Main Part
\section{Résultats}\label{sec:resultats}
%	Cette section doit contenir les résultats que vous avez obtenu
%	avec la méthodologie décrite dans la section \ref{sec:met}.
%	Les résultats devront être présentés de préférence sous forme de tableau
%	(cf. Table~\ref{tab:simParameters})
Pour réaliser nos tests, nous avons réduit l'aléatoire au maximum en
	 assignant une population initiale commune PI à toutes les
		simulations, d'elle découle population\_size
\subsection{Paramètrage minimal}\label{subsec:parametrage-minimal}
	Pour évaluer notre paramétrage, nous le comparons tout d'abord
	à un paramétrage minimal, c'est-à-dire avec les paramètres par
	défaut de PyGAD augmentés du minimum requis.
%\subsubsection{Paramètres par défaut de PyGAD}%\label{subsec:parametres-par-defaut-de-pygad}
	\begin{table}[!hbt]
		% Center the table
		% Title of the table
		\caption{Paramètres par défaut de PyGAD}
		\label{tab:PyGAD_parameters_default}
		% Table itself: here we have two columns which are centered and have lines to the left, right and in the middle: |c|c|
		\begin{tabular}{|l|l|}
			\hline
			\textbf{Parameter} & \textbf{Value} \\
			\hline
			gene space & [-1, 1] \\
			\hline
			population size & 50 \\
			\hline
			num parents mating & 4 \\
			\hline
			parent selection type & random \\
			\hline
			crossover type & single point \\
			\hline
			mutation type & random \\
			\hline
			mutation percent genes & 10 \\
			\hline
			random mutation min val & -1 \\
			\hline
			random mutation max val & 1 \\
			\hline
			\end{tabular}
	\end{table}

%\subsubsection{Paramètrage minimum requis}%\label{subsec:parametres-minimum-requis}
	\begin{table}[!hbt]
		% Center the table
		\begin{center}
		% Title of the table
		\caption{Paramètrage minimum requis}
		\label{tab:Parametrage_minimum_requis}
		% Table itself: here we have two columns which are centered and have lines to the left, right and in the middle: |c|c|
		\begin{tabular}{|l|l|}
			\hline
			\textbf{Parameter} & \textbf{Value} \\
			\hline
			num genes & 8 joints * 500 time steps = 4000 \\
			\hline
			initial population & PI \\
			\hline
			num parents mating & (population\_size // 4) + 2 \\
			\hline
			fitness function & Equation\ref{eq:equation} \\
			\hline
			num generations & 100 \\
			\hline
			\end{tabular}
		\end{center}
	\end{table}
	Les résultats obtenus sont présentés dans la figure\ref{fig:parameter_minimum_requis}
	et la figure\ref{fig:parameter_minimum_requis2}.
	Nous remarquons que les dominants présentant une avancée majeure
	ne semblent pas tirer les autres vers le haut.
	Le nombre d'individus avec lesquels ils se reproduisent
	est trop faible. Leur génome est ensuite muté au même taux que celui
	des dominés.

% This is how you define a table: the [!hbt] means that LaTeX is forced
% 	(by the !) to place the table exactly here (by h),
% 	or if that doesnt work because of a pagebreak or so, it tries to
% 	place the table to the bottom of the page (by b) or the top (by t).

\subsection{Importance de la diffusion du savoir}
	Nous ajoutons le parametrage suivant (table\ref{tab:simParameters}):
	\begin{table}[!hbt]
		% Center the table
		\begin{center}
		% Title of the table
		\caption{Paramètrage pour la diffusion du savoir}
		\label{tab:simParameters}
		% Table itself: here we have two columns which are centered and have lines to the left, right and in the middle: |c|c|
		\begin{tabular}{|l|l|}
			\hline
			\textbf{Parameter} & \textbf{Value} \\
			\hline
			num parents mating & (population size // 2) + 2 \\
			\hline
			parent selection type & tournament \\
			\hline
			K tournament & population size // 2 + 2 \\
			\hline
			keep elitism & 2 \\
			\hline
			mutation type & adaptive \\
			\hline
			mutation percent genes & (30, 10) \\
			\hline
			\end{tabular}
		\end{center}
	\end{table}

%\newpage

	Nous obtenons les résultats suivants (figure\ref{fig:300tournament14elit2} et figure\ref{fig:300tournament14elit2()}):
\begin{figure}[!hbt]
  \centering
  \includegraphics[width=\columnwidth]{image/pop50=default/100gen/initialpop375/300-10-10}
  \caption{fitness par génération pour un paramétrage minimal}
  \label{fig:parameter_minimum_requis}
\end{figure}

\begin{figure}[!hbt]
  \centering
  \includegraphics[width=\columnwidth]{image/pop50=default/100gen/initialpop375/300-10-10()}
  \caption{plages de valeurs par génération pour un paramétrage minimal}
  \label{fig:parameter_minimum_requis2}
\end{figure}

\begin{figure}[!hbt]
%  \includegraphics[width=\columnwidth]{image/pop50=default/100gen/initialpop375/300tournament14elit2}
  \includegraphics[width=\columnwidth]{image/pop50=default/100gen/initialpop375/500}
  \caption{fitness par génération pour notre paramétrage}
  \label{fig:300tournament14elit2}
\end{figure}

\begin{figure}[!hbt]
  \includegraphics[width=\columnwidth]{image/pop50=default/100gen/initialpop375/500()}
  \caption{plages de valeurs par génération pour notre paramétrage}
  \label{fig:300tournament14elit2()}
\end{figure}
%	et correctement référencés.
%	Les conditions d'expérimentation
%	(c-à-d matériel et logiciels utilisés) devront être ainsi indiquées.
%	En plus des résultats mêmes, cette section devra contenir
%	votre propre analyse et discussion de résultats
%	(par exemple comparaison par rapport à une méthode de référence)

%\newpage

\section{Discussion}
	Nous pouvons observer une nette amélioration de la qualité des solutions
	grâce à notre paramétrage.
	La progression faite à la génération 40 est rapidement répandue
	à l'ensemble de la population jusqu'à la génération 70.
	Cependant, la borne inférieure de la fitness générale
	correspondante n'évolue pas en conséquence.
	Cela traduit une sous-estimation de la distance parcourue par
	rapport aux autres composante, notamment la vitesse.
	Les bonnes solutions étant indiscernables des mauvaises par leur fitness,
	la progression sur la distance s'annule à la génération 70 et
	s'inverse	par	la
	suite.
	La borne supérieure quand à elle, atteint un plafond qui semble
	impossible à traverser puis chute brutalement à la dérnère génération.
	L'extremité supérieure de la plage de valeurs représentant
	les individus dominants, son immuabiblité est due à leur taux
	de mutation trop faible.

	Nous avons également pu constater qu'il est très difficile
	de trouver les bons paramètres pour que l'algorithme converge
	vers une solution optimale.
	Même sur une population initiale commune et un paramétrage identique
	, la qualité des solutions trouvées est très variable.
%	De plus, il est très difficile
%	de trouver une fonction de fitness qui permette de bien
%	évaluer les solutions.
	Nous pouvons cependant affirmer certains points:

	La mutation doit opérer prioritairement sur les dominés mais
	ne doit pas épargner les dominants ou un maximum local est atteint.
	Des valeurs de mutation aléatoires trop grandes reviennent à
	ignorer les parents et par là même, l'évolution par transmission
	des gènes.
	Celles-ci doivent donc être comprises dans un
	intervalle autour de zéro très restreint.

	Si l'élitisme est trop important ou la compétition est trop sélective,
	les solutions
	ne sont pas assez diversifiées.
	Si la mutation est trop forte, les solutions convergent trop rapidement.
	Lorsqu'une ou plusieurs de ces conditions sont réunies, l'évolution
	atteint un maximum local et la progression est fortement ralentie.
	Ce phénomène se traduit visuellement par les plateaux observés.

\section{Conclusion}\label{sec:conclusion}
%Cette section contient un rappel des contributions
%/ de résultats importants de votre article
	Notre simulateur permet de trouver en temps raisonnable des
	solutions permettant à des créatures de se maintenir debout sans
	chuter et se déplacer le temps de la simulation.

	De tout système, ce que nous attribuons au hasard est sa partie
	incomprise.
	Dans notre simulateur, un système quasi-totalement compris,
	l'aspect stochastique	de
	l'algorithme génétique est donc fondamentalement sous-optimal.
	Cependant, nous le contrôlons très peu, et même si nous en avions
	la
	complète maîtrise, la taille de l'algorithme déterministe optimal
	serait (beaucoup) plus grande que celle de l'algorithme génétique.

	L'algorithme génétique nous semble donc être le parfait compromis
	entre les objectifs suivants :
	\begin{itemize}
		\item algorithme compact
		\item adaptabilité au type d'information disponible
		\item robustesse
		\item exhaustivité des solutions (fonction du temps d'execution)
	\end{itemize}

	Les valeurs d'entrée ont un impact minime sur la qualité
	des solutions trouvées.

\subsection{Limites}\label{subsec:limites}
	Les limites de notre simulateur sont les suivantes :
	\begin{itemize}
		\item Les paramètres par défaut de PyGAD ont à peine été
		modifiés. La vision globale du traitement de la problématique
		offerte par l'espace de notre parametrisation est très étroite.
		\item La majeure partie du temps d'execution du programme est
		consacrée à la simulation physique dont nous ne valorisons que
		très peu les informations que nous en retirons.
		\item Il est difficile de déterminer si la fonction de
		fitness
		est trop sévère par rapport au principe stochastique de l'algorithme
		ou trop laxiste vis-à-vis de la simulation physique.
	\end{itemize}
	Les forces à appliquer sur les articulations étant inscrites dans
	le génome, et la créature étant dépourvue de sens, aucune adaptation
	intra-générationelle n'est possible.
	La moindre irrégularité introduite dans l'envirronement
	pourrait faire chuter la créature.
%et éventuellement une indication sur les perspectives de recherche future dans le même domaine.
\subsection{Perspectives de recherche future}\label{subsec:perspectives-de-recherche-future}
	L'ensemble des idées présentées dans cette section ont été imaginées
	avant d'en avoir trouvé les implémentations présentées dans la
	section\ref{sec:etat-de-l'art}. Elles s'avèrent toutes se situer
	parmi les solutions les plus efficaces, nous pouvons donc affirmer
	que ce projet nous a permis de cerner les aspects les plus critiques
	de la méthode utilisée.

\subsubsection{Paramètres}
	L'étendue des paramètres de PyGAD est très large et au vu de nos
	résultats obtenus après les tests présentés,
	une solution
	optimale est probablement accessible en temps raisonnable via
	la bonne combinaison de paramètres.
	Pour la trouver, nous cherchons actuellement un moyen d'impl
	émenter un \textbf{hyperparamétrage}, l'idée serait	de nous
	épargner
	une batterie de
	tests interminable en
	utilisant du scrapping pour récupérer les paramétrisations d'un
	maximum de
	projet similaires au notre utilisant PyGAD et comparer leurs
	résultats afin de
	trouver
	les
	meilleurs paramètres puis d'itérer sur cette base pour affiner
	les paramètres dont les valeurs sont des intervalles. Il s'agirait
	pour ainsi dire d'un meta algorithme génétique appliqué sur
	la population des param
	ètres.

	\textbf{Eviter les maximum locaux} semble primordial et pourrait
	se faire en remplaçant les paramètres
	d'élitisme et de compétition par des intervalles contrôlés par une
	fonction réagissant à la stagnation de la population.

\subsubsection{Bibliothèque}
	L'information de marche n'est actuellement portée que par la
	population vivante à un instant donné. L'intérêt d'une population
	vivante est qu'elle expérimente. L'intérêt d'une bibliothèque
	est qu'elle conserve l'information.
	Il serait	intéressant de permettre à la population de se
	renseigner sur les solutions notables trouvées par des individus
	qui lui sont antérieurs de plus d'une génération, afin de muter
	en fonction de ces informations.
	Cette bibliothèque serait alimentée par les solutions que nous
	sauvegardons
	déja pour la post-visualisation ou l'assignation du paramètre
	\texttt{initial\_population}.
	Elle remplirait le rôle déjà joué par la population élite, mais
	ne présenterait pas les défauts de mutation à taux réduit et
	consanguinité.
	Contrairement à la population élite qui partage son information à
	des individus en quantité limitée, et l'impose, la bibliothèque,
	puisqu'accessible à tous les individus, permettrait une diffusion
	du savoir bien plus large et harmonieuse chacun libre d'en
	assimiler
	ce qui lui convient.

	\textbf{Un dictionnaire de génomes à éviter}, tels qu'une suite
	de zéros
	entraînant l'absence d'action, et/ou assurément trop loin d'une
	solution viable par rapport au nombre de générations restantes.

\subsubsection{Démocratie}
	Afin d'éviter le phénomène d'extremum local, nous pourrions limiter
	la quantité d'individus se spécialisant vers une solution
	particulière. La diversité serait ainsi toujours majoritaire.

\subsubsection{Fonction de fitness}
%	\textbf{correction} d'un mauvais comportement, en plus de
%	l'attribution	de score, nous
%	pourrions implémenter une fonction de correction pour identifier
%	la région du génome commandant une action non désirée et la
%	remplacer par des valeurs plus adaptées.

	La force de l'algorithme génétique étant sa souplesse, il est possible
	que des conditions plus laxistes permettent d'éviter de rester coincé
	dans un maximum local.
	L'objectif de hauteur du centre de masse est
	actuellement un maximum en un point et peut entraîner un dés
	équilibre limitant les possibilités de mouvement.
	Cet objectif pourrait être remplacé par	un intervalle de hauteur
	global allouant plus de liberté à la créature.

	La fonction de fitness pourrait elle-même être modifiée
	automatiquement en fonction des comportements observés en liant
	les poids de chaque composante aux actions pour lesquelles elles
	entrent en jeu.

\subsubsection{Adaptation intra-générationelle}
	\textbf{Réseau neuronal}
	Ces perspectives de développement de la fonction de fitness gén
	érationnelle
	mènent rapidement aux prémices d'un réseau neuronal.
	Lié à des capteurs, il permettrait à la créature de réagir
	à son environnement et de se déplacer de manière autonome.
	Il s'agirait dans notre cas d'une forme d'\textbf{épigenetique}
%
\subsubsection{Environnement}
	Dès lors que la créature est capable de réagir à son environnement,
	celui-ci peut être modifié pour augmenter la difficulté.
	Parmi les modifications les plus intéressantes :
	\textbf{Générales}:
	\begin{itemize}
			\item pentes
			\item obstacles
			\item projectiles
			\item créatures adverses
		\end{itemize}

	\textbf{A finalité réelle}:
	Se concentrer sur les surfaces encore mal appréhendées par les
	robots actuels:
	\begin{itemize}
			\item à solidité variable ( boue, etc.)
			\item glissantes ( verglas, etc.)
			\item instables ( sable, etc.)
		\end{itemize}

	\textbf{A finalité virtuelle}:
	\\Environnements de jeux vidéo déjà existants

%	Ce rapport est un écrit original dans la mesure où toute idée
%	apportée est soit originale, soit trop transformée soit documentée
%L'écriture du code et du rapport a été accélérée (et non générée) à
%	l'aide d'un correcteur orthographique et du
%plugin \href{https://copilot.github.com/}{Github Copilot}
%
%La recherche relative à l'écriture du rapport a été accélérée à l'aide
%de \href{https://chat.openai.com/?model=text-davinci-002-render-sha}{ChatGPT 3.5}
%	La recherche de l'état de l'art
%\bibliographystyle{unsrt}
%\bibliography{bibliography}
%\bibliography{References}
\begin{thebibliography}{99}
	\bibitem{replicating-automata} John von Neumann
	\textit{Theory of Self-Reproducing Automata}, 1966.
	\bibitem{near-term_self-replicating} Olivia Borgue et Andreas M.
	Hein \textit{Near-Term Self-replicating Probes - A Concept Design}, 2005.
	\bibitem{systems-adaptation} John Holland \textit{Adaptation in
	Natural
	and artificial Systems}, 1975.
	\bibitem{Cormier} Gabriel Cormier \textit{Systèmes Intelligents}, Université de Moncton, 2019.
	\bibitem{MOEA/D} Qingfu Zhang and Hui Li \textit{MOEA/D: A Multiobjective Evolutionary Algorithm Based on Decomposition}, 2008.
	\bibitem{Sims} Karl Sims \textit{Evolving Virtual Creatures}, 1994.
	\bibitem{Graham,Lee;Oppacher,Franz} Graham, Lee; Oppacher, Franz. \textit{
	Speciation Through Selection and Drift. } Proceedings of The Eleventh IASTED International Conference on Artificial Intelligence and Soft Computing. ACTA Press.
	\bibitem{Josh-C-Bongard} Josh C Bongard \textit{The legion
	system: A novel approach to evolving heterogeneity for collective problem solving}.
	\bibitem{extrema-global-local} By KSmrq -  \url{http://commons.wikimedia.org/wiki/File:Extrema_example.svg, GFDL 1.2, https://commons.wikimedia.org/w/index.php?curid=6870865}
	\bibitem{pygad} \url{https://pygad.readthedocs
	.io/en/latest/}

	\bibitem{quantum-computing} , Bart Rylander, Terence Soule,
	James Foster and Jim Alves-Foss \textit{Quantum Genetic Algorithms}, 2000
\end{thebibliography}

%
%\newpage
%
%\appendices
%\section{Consignes}\label{sec:consignes}
%% Main Part
%\subsection*{Document}
%	% LaTeX takes complete care of your document layout ...
%	Le rapport doit être rédigé en \LaTeX{} en utilisant ce template.
%	La longueur du rapport ne devra pas, en tout cas, dépasser les 6 pages.
%	Ce rapport doit être \emph{self-contained}, c-à-d il doit pouvoir être lu et compris sans avoir besoin de se documenter ailleurs.



% Your document ends here!
\end{document}
